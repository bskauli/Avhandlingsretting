\title{The Griffiths Group of 1-cycles on Double Covers}
\author{Bjørn Skauli}
\date{}
\maketitle
\label{pap:griffiths}

\begin{abstract}
  We show that the Griffiths group of 1-cycles is trivial on double covers of sufficiently low degree by adapting the technique used by Tian and Zong in \cite{TZ} to prove the same result for complete intersections. Using a different technique, Minoccheri and Pan have obtained the same result in \cite{MP}.
\end{abstract}
%Introduction parts:
%Voisin conjectures that Griffiths group trivial for Fano varieties
%Work of Tian-Zong
%Work of Minoccheri-Pan
%Work over C in fact, due to the Taylor series argument
%Acknowledgements
\section{Introduction}
% A major goal in birational geometry is finding birational invariants that distinguish rational and irrational varieties. Building on this, we also want invariants that distinguish between the various weaker rationality properties. For any given birational invariant, this leads to asking whether rational connectedness implies triviality of the invariant, since rational connectedness is one of the weakest rationality properties. Since a smooth Fano variety is rationally connected, a closely related question is if the given invariant is trivial on all smooth Fano varieties.

In this paper, we will consider the Griffiths group of 1-cycles, $\Griff_1(X)$, of a variety $X$. This group is defined as the quotient $Z_1(X)_{hom}/Z_1(X)_{alg}$, where $Z_1(X)_{hom}$ and $Z_1(X)_{alg}$ denote the subgroup of homologically and algebraically trivial 1-cycles, respectively. This quotient group is a birational invariant of smooth projective varieties.

In Griffiths' original paper \cite{GriffithsPeriodsRational}, it is proven that for $X$ a general complex quintic threefold, $\Griff_1(X)$ is an infinite group. Later, Clemens proved in \cite{ClemensNotFinitelyGenerated} that in fact $\Griff_1(X)$ is infinitely generated, even modulo torsion. Later, Voisin proved that for any Calabi-Yau threefold $X$ with $h^1(T_X) \neq 0$, $\Griff_1(X)$ is infinitely generated, even mod torsion.
%\todo{Add something about higher Griffiths groups and relation to unramified cohomology?}
These examples all have trivial canonical divisor, and are thus clearly not rational.

Conversely, it is an interesting question to study which varieties have trivial first Griffiths group of 1-cycles. Using a decomposition of the diagonal with $\Q$-coefficients, Bloch and Srinivas \cite{BSCorrespondences} prove that if $CH_0(X) \otimes \Q$ is universally supported on a surface,
algebraic and homological equivalence coincide for codimension 2 cycles. Recall that for a variety $X$ we say that $\CH_0(X) \otimes \Q$ is universally supported on a surface if there exists a surface $V \subset X$, such that $\CH_0((X \setminus V) \times F) \otimes \Q = 0$ for any extension $F$ of the ground field.
As a consequence of Bloch and Srinivas' result, any smooth, projective, rationally connected complex variety of dimension 3 has trivial Griffiths group of 1-cycles.

For rationally connected varieties of any dimension, Voisin proves the following about $\Griff_1$.
\begin{theorem}[{\cite[Lemma 2.23]{VoisinLueroth}}]
	If $X$ is smooth, projective and rationally connected, the group $\Griff_1(X)$ is a torsion group.
\end{theorem}

 Voisin has raised the question of whether $\Griff_1(X)$ is trivial for all rationally connected varieties. An important class of examples of such varieties is Fano complete intersections in projective space. In light of Voisin's question, it is interesting to check whether $\Griff_1$ is trivial for these complete intersections. In \cite{TZ}, Tian and Zong make great progress towards this by proving that for a Fano complete intersection $X$ in projective space of index at least 2, $\Griff_1(X)$ is trivial.
 
 Recall that the index of a Fano variety $X$ with Picard number 1 is the largest integer $\iota$ such that $-K_X = \iota H$ for some ample divisor $H$. For many purposes, one can think of the index as one measure of how close the variety is to $\P^n$. In fact, if the index of a smooth Fano variety is greater than its dimension, then it is isomorphic to projective space (\cite{KO}).

The main theorem in \cite{TZ} is the following:
\begin{theorem}[{\cite[Theorem 1.3]{TZ}}]
\label{thm:TZEquivalentToRationalCurves}
  Let $X$ be a smooth, proper and separably rationally connected variety over an algebraically closed field. Then every 1-cycle is rationally equivalent to a $\Z$-linear combination of the cycle classes of rational curves. That is, the Chow group $\CH_1(X)$ is generated by rational curves.
\end{theorem}
Using this result, Tian and Zong prove that $\Griff_1(X)$ is trivial. The main step is proving the following result.
\begin{theorem}[{\cite[Theorem 6.2]{TZ}}]
\label{thm:TZ}
  Let $X$ be a (possibly singular) complete intersection of type $(d_1,\dots,d_c)$, with $d_1 + \cdots + d_c \leq n-1$, in $\P^n$. Then every rational curve on $X$ is algebraically equivalent to a union of lines.
\end{theorem}
For a Fano smooth complete intersections of dimension at least 3, the Fano scheme of lines is connected, hence any two lines are algebraically equivalent. Since $H_2(X,\Z) \simeq \Z$, we have the following immediate corollary.
\begin{theorem}[{\cite[Remark 6.4]{TZ}}]
  Let $X \subset \P^n$ be a smooth complete intersection of type $(d_1,\dots,d_c)$, with $d_1 + \cdots + d_c \leq n-1$. Then $\Griff_1(X) = 0$.
\end{theorem}

 In \cite{MP}, Minoccheri and Pan study 1-cycles in the more general setting of Fano complete intersections in weighted projective space. Their approach is based on the evaluation maps associated to the Kontsevich space of stable maps. For complete intersections of sufficiently low degree, they prove that the Griffiths group of 1-cycles is trivial. Their main result on the Griffiths group of 1-cycles is:
\begin{theorem}[{\cite[Theorem 2.3]{MP}}]
\label{thm:MP}
  Let $X \subset \P(a_0,\dots,a_n)$ be a smooth weighted complete intersection of $c$ hypersurfaces of degrees $d_1, \dots, d_c$. If the following conditions hold,
  \begin{enumerate}[i)]
  \item $\dim X \geq 3$
  \item $a_1 = a_2 = a_3 = 1$
  \item $a_3 + \cdots + a_n + 2 + c-n \leq d_1 + \cdots + d_c$
  \item $d_1 + \cdots + d_c \leq a_3 + \cdots + a_n$,
  \end{enumerate}
 and 
\[\iota(X) > \frac{1}{2} \dim(X), \]
then $\Griff_1(X) = 0$.
\end{theorem}
In particular, for smooth double covers $X$ of $\P^n$ branched along a divisor of degree $2d$, \cref{thm:MP} implies that $\Griff_1(X) = 0$ if $d$ is less than $\frac{n}{2}$, which is precisely half of the Fano bound.

In \cite{MP}, Minoccheri and Pan compare the bound on the index in \cref{thm:MP} to the one given in \cref{thm:TZ} and ask what results an application of the techniques of Tian-Zong to the weighted projective case would give. The goal of this paper is to apply the technique of Tian-Zong to the case of double covers. Adapting the techniques in \cite{TZ} to double covers, we obtain the following result, which is the same result as Minoccheri and Pan.
\begin{theorem}
  \label{thm:GriffTrivialIntro}
  Let $p \from X \to \P^n$ be a smooth double cover branched over a hypersurface of degree $2d$, where $d < \frac{n}{2}$. Then $\Griff_1(X) = 0$.
\end{theorem}
The reason the technique does not give us better results is the inductive argument we use in the proof of \cref{lem:RationalCurves2Lines}. It is possible that through a more careful analysis, one could improve on the bound in \cref{thm:GriffTrivialIntro}.

We will work over $\C$ throughout and use the notations and definitions for double covers from \cref{pap:linesondoublecovers}.




% \section{Preliminaries}

% \subsection{The Griffiths Group}
% \todo{Define groups of 1-cycles?}
% \begin{definition}
%   Let $X$ be a projective complex variety. $Z_1(X)$ is the free abelian group generated by irreducible curves on $X$. We write $Z_1(X)_{hom}$ for the cycles that are homologically equivalent to 0, and $Z_1(X)_{alg}$ for the cycles that are algebraically equivalent to 0.
% \end{definition}
% \begin{definition}
%   \label{def:Griffiths}
%   The \emph{first Griffiths group} $\Griff_1(X)$ is defined as $Z_1(X)_{hom}/Z_1(X)_{alg}$.
% \end{definition}
% \begin{theorem}[{\cite[Proposition 1.30]{VoisinDoD}}]
%   \label{thm:GriffInvariant}
%   If two smooth projective varieties $X$ and $Y$ are stably birational, then $\Griff_1(X) \simeq \Griff_1(Y)$.
% \end{theorem}
% For rationally connected varieties, so in particular Fano varieties, we have the following result on $\Griff_1(X)$.
% \begin{theorem}[{\cite[Lemma 2.23]{VoisinLueroth}}]
% 	If $X$ is smooth, projective and rationally connected, the group $\Griff_1(X)$ is a torsion group.
% \end{theorem}

%Definition of double covers
%Definition of the Griffiths group
%Definition of $\Mor_e(X,Y)$
%Reference to Kontsevich space of stable maps? (Debarres notes covers morphism space at least)

\section{1-Cycles on Double Covers of Low Degree}
\cref{thm:TZEquivalentToRationalCurves} also applies to Fano double covers. To deduce \cref{thm:TZ} from \cref{thm:TZEquivalentToRationalCurves}, Tian and Zong prove that on a Fano hypersurface in $\P^n$ of index at least 2, or equivalently degree $\leq n-1$, any rational curve is algebraically equivalent to a union of lines. To study double covers with the same technique, our main task is therefore to show the following result:
\begin{lemma}
  \label{lem:RationalCurves2Lines}
  Let $p \from X \to \P^n$ be a double cover branched over a hypersurface of degree $2d$, where $d < \frac{n}{2}$. Then any rational curve $C \subset X$ is algebraically equivalent to a union of lines.
\end{lemma}

Before we prove \cref{lem:RationalCurves2Lines}, let us see how it implies the main result.
\begin{theorem}
  \label{thm:GriffTrivial}
  Let $p \from X \to \P^n$, with $n \geq 3$, be a smooth double cover branched over a hypersurface of degree $2d$, with $d < \frac{n}{2}$. Then $\Griff_1(X) = 0$.
\end{theorem}
\begin{proof}
  Fix a line $l_0 \subset X$. The class $[l_0] \in H_2(X,\Z)$ is a generator of this homology group, by the Lefschetz hyperplane theorem. It follows that any curve $C \subset X$ is homologically equivalent to $e[l_0]$, so $Z_1(X)_{hom}$ is generated by differences of the form $[C]-e[l_0]$. We must show that $C$ is also algebraically equivalent to $e[l_0]$, and therefore $[C]-e[l_0]$ is also contained in $Z_1(X)_{alg}$ for any curve $C$. By \cref{thm:TZEquivalentToRationalCurves} it suffices to prove that any rational curve on $X$ of degree $e$ is algebraically equivalent to $e[l_0]$. Assuming, for the moment, that \cref{lem:RationalCurves2Lines} holds, $C$ is algebraically equivalent to a sum $\sum_{i=1}^e[l_i]$, where the $l_i$ are lines on $X$. Since $X$ has dimension at least 3, and the degree of $X$ is less than $2n-3$, the space of lines on $X$ is connected by \cref{prop:FConnected}, so $[l_i]$ and $[l_0]$ are algebraically equivalent for all $i$, and we are done.
\end{proof}

\begin{remark}
The assumption that $\dim X \geq 3$ is not very restrictive, since for any $X$ of dimension 1 or 2, $\Griff_1(X) = 0$. On the other hand, the assumption $d \leq \frac{n}{2}$ is quite restrictive. In particular, it means that in any given dimension, only half of the Fano double covers in that dimension are covered by \cref{thm:GriffTrivial}.
\end{remark}


We will prove \cref{lem:RationalCurves2Lines} following a strategy similar to the one used in \cite[Theorem 6.2]{TZ}. However, using the relevant connectedness result, \cref{lem:ProjectiveConnectedness}, is a more intricate process on double covers.

Recall that for a projective variety $X$, the space $\Mor_e(\P^1,X)$ parametrizes morphisms of degree $e$ from $\P^1$ to $X$ and is equipped with a universal morphism. Maps $\P^1 \to \P^n$ are parametrized by an $(n+1)$-tuple of polynomials in $H^0(\P^1,\sO_{\P^1}(e))$, so one compactification of $\Mor_e(\P^1,X)$ is $\P^{(n+1)(e+1)-1}=\P^{ne+n+e}$. If $X \subset \P^n$ is a subvariety, then $\Mor_e(\P^1,X)$ is a subvariety of $\Mor_e(\P^1,\P^n)$. It can therefore be compactified by a subvariety of $\P^{ne+n+e}$. While this compactification is useful to apply \cref{lem:ProjectiveConnectedness}, we will also at one point use the Kontsevich space of stable maps, where also the boundary points of the compactification correspond to morphisms, but from reducible domains. A reference for the Kontsevich space of stable maps is \cite{FP}.

As the first step, we consider when a morphism in $\Mor_e(\P^1, \P^n)$ lifts to a morphism in $\Mor_e(\P^1,X)$ \ie a map to the double cover $p \from X \to \P^n$. Let $\phi \from \Mor_e(\P^1,X) \to \Mor_e(\P^1,\P^n)$ be the map induced by composition with $p$.
\begin{lemma}
\label{lem:RationalCurveLift}
  Let $p \from X \to \P^n$ be a double cover with branch divisor $B \subset \P^n$ defined by $F(x_0,\dots,x_n) = 0$ for a polynomial $F$ of degree $2d$. Then the image of $\phi$ in $\Mor_e(\P^1,\P^n)$, is precisely the curves
\[(z_0,z_1) \mapsto (g_0(z_0,z_1),\dots,g_n(z_0,z_1)),\]
such that $F(g_0(z_0,z_1),\dots,g_n(z_0,z_1)) = (h(z_0,z_1))^2$ for some polynomial $h(z_0,z_1)$ of degree $de$.
\end{lemma}
\begin{proof}
  The double cover $X$ is defined in the weighted projective space $\P(1,\dots,1,d)$ by 
\[y^2 - F(x_0,\dots,x_n) = 0.\]
 Assume that $(z_0,z_1) \mapsto (g_0(z_0,z_1),\dots,g_n(z_0,z_1),h(z_0,z_1))$ is a parametrized curve in $\Mor_e(\P^1,X)$, where the $g_i$ have degree $d$ and $h$ has degree $de$. Since the curve lies in $X$, 
\[h(z_0,z_1))^2 - F(g_0(z_0,z_1),\dots,g_n(z_0,z_1) = 0,\]
or equivalently 
\[F(g_0(z_0,z_1),\dots,g_n(z_0,z_1)) = (h(z_0,z_1))^2.\]
Conversely, let 
\[ f \from (z_0,z_1) \mapsto (g_0(z_0,z_1),\dots,g_n(z_0,z_1))\]
 be a parametrized curve in $\Mor_e(\P^1,\P^n)$, such that 
\[F(g_0(z_0,z_1),\dots,g_n(z_0,z_1)) = (h(z_0,z_1))^2.\] Then 
$(g_0(z_0,z_1),\dots,g_n(z_0,z_1),h(z_0,z_1))$ is a curve in $\Mor_e(\P^1,X)$ mapping to $f$ when composed with $p$.
\end{proof}

Next we turn to understanding the locus of morphisms  $(g_0(z_0,z_1),\dots,g_n(z_0,z_1))$, such that $F(g_0(z_0,z_1),\dots,g_n(z_0,z_1))$ is a square. The following two lemmas will give a partial description of this locus by the vanishing of polynomials.

Note that the map $\P(\sO_{\P^1}(m)) \to \P(\sO_{\P^1}(2m))$ given by squaring the polynomial is algebraic, so its image is a projective variety. But later it will  be important to control how many polynomials are necessary to generate the ideal of the image, at least on an open subset. For this we have the following result.

\begin{lemma}
	\label{lem:SquareCondition}
  Let $d$ be a positive integer, and let $f(x) = \sum_{i=0}^{2d}a_ix^i$ be a polynomial in a single variable of degree $2d$. Assume that the constant term $a_0$ of $f$ is nonzero. Then there are $d$ polynomials $P_1,\dots,P_d$, in the coefficients $a_i$ of $f,$ such that
\[P_1(a_1,\dots,a_{2d}) = \cdots = P_d(a_1,\dots,a_{2d}) = 0\]
if and only if $f(x) = (g(x))^2$ for some polynomial $g$ of degree $d$.
\end{lemma}
\begin{proof}
  Since $a_0 \neq 0$, $\sqrt{f(x)}$ is a holomorphic function in a neighborhood of $x=0$, it can be represented as a power series $\sum_{i=0}^\infty b_ix_i$ in $x$. The polynomial $f(x)$ admits a square root $g(x)$ precisely when this power series is a polynomial of degree $d$.

We use the equality
\[(\sum_{i=0}^\infty b_ix_i)^2 = \sum_{i=0}^{2d}a_ix^i \]
to express the coefficients $b_i$ in terms of the $a_i$. Pick $b_0$ such that $b_0^2 = a_0$. By comparing terms of degree 1, we find that $2b_0b_1 = a_1$, or equivalently $b_1 = \frac{a_1}{2b_0}$. Comparing terms of degree 2 gives the equality $2b_0b_2 + b_1^2 = a_2$. After inserting the expression for $b_1$ and reorganizing, we get
\[b_1 = \frac{2a_0a_2 - a_1^2}{(2a_0)^2}.\]

In general, by comparing coefficients we get
\[ b_k = \frac{a_k - \sum_{i+j = k}b_ib_j}{2b_0}.\]
%\todo{Add reference to StackOverflow?}
By recursively replacing the $b_i$ for $0<i<k$, one then obtains an expression for $b_k$ in terms of the coefficients $a_i$ of $f$, with a power of $b_0$ in the denominator.

Now assume $b_k = 0$ for all $d < k \leq 2d$. Since $a_k = 0$ for $2d<k$, one sees from the recursive expression for $b_k$ that necessarily $b_k=0$ for all $k>d$. Hence $f(x)$ admits a polynomial square root $g(x)$. The condition $b_k = 0$ for $d<k\leq 2d$ can be expressed as the vanishing of $d$ polynomials $P_1,\dots,P_d$ in the coefficients $a_i$ of $f$.
\end{proof}
\begin{remark}
  By multiplying by an appropriate power of $a_0$, one may construct polynomials in the $a_i$ whose simultaneous vanishing imply that either $\sum_{i=0}^{2d}a_ix^i$ is a square, or $a_0 = 0$. Homogenizing the polynomials in \cref{lem:SquareCondition} shows that an analogous conclusion holds for homogenous polynomials in two variables.
\end{remark}

A dimension count shows that the codimension of polynomials of degree $2d$ admitting a polynomial square root is $d$, so the locus of square polynomials must be defined by at least this many polynomials.

For degree 2 polynomials, there is a well-known simple condition for when it is a square, which applies globally.
\begin{lemma}
\label{lem:QuadricSquarecondition}
	A complex polynomial $a_0 + a_1x + a_2x^2$ of degree $2$ is of the form $(b_0 + b_1x)^2$ if and only if 
	\begin{equation}
		\label{eq:QuadricSquareCondition}
		4a_0a_2 - a_1^2 = 0.
	\end{equation}
\end{lemma}
\begin{proof}
	It is straightforward to check that any polynomial of the form $(b_0 + b_1x)^2$ satisfies \eqref{eq:QuadricSquareCondition}. Conversely, if the polynomial $a_0 + a_1x + a_2x^2$ satisfies \eqref{eq:QuadricSquareCondition}, then either $a_1$ and one of $a_0$ and $a_2$ are equal to zero, in which case the polynomial is a square. Or, all coefficients are nonzero,  so if $b_0$ is a square root of $a_0$, and $b_1$ a square root of $a_2$, then $(b_0 + b_1x)^2 = b_0^2 + 2b_0b_2x + b_2^2x^2 = a_0 + 2b_0b_1x+a_2x^2$. From \eqref{eq:QuadricSquareCondition} it follows that $(2b_0b_1)^2 = a_1^2$, so $a_1 = \pm 2b_0b_1$. Hence, after possibly replacing $b_1$ with the other square root of $a_2$, we find that $(2b_0b_1)=a_1$, so $a_0 + a_1x + a_2x^2 = (b_0 + b_1x)^2$.
\end{proof}

We will need the following lemma about connectedness of subvarities in projective space, which can be found in \cite[Expos\'e XIII, (2.1) and (2.3)]{SGA2}. This lemma is a crucial part of both our proof of \cref{lem:RationalCurves2Lines} and Tian and Zong's proof of \cref{thm:TZ}.
\begin{proposition}
	\label{lem:ProjectiveConnectedness}
	Let $X$ be a subscheme in $\P^N$ defined by $M$ homogenous polynomials. Let $Y$ be a closed subset of $X$ of dimension less than $N-M-1$. Then $X \setminus Y$ is connected.
\end{proposition}

We will also have use for a simple topological lemma.
\begin{lemma}
  \label{lem:Topological}
	Let $X,Y \subset Z$ be closed subspaces of a topological space. Assume that $X \union Y$ and $X \cap Y$ are both connected and $X \cap Y$ is nonempty. Then $X$ and $Y$ are connected.
\end{lemma}
\begin{proof}
	The statement is symmetric in $X$ and $Y$, so it suffices to prove it for $X$. Let $X_1, X_2 \subset X$ be closed disjoint sets, such that $X = X_1 \cup X_2$. Since $X \cap Y$ is connected, the intersection must lie wholly in one of the two closed sets, say $X_1$. Now $(Y \cup X_1)$ and $X_2$ are two disjoint closed sets whose union is $X \union Y$. Since this latter space is connected and $Y \cap X_1$ is nonempty, we must have $X_2 = \emptyset$, hence $X$ is connected.
\end{proof}

We are now ready to prove \cref{lem:RationalCurves2Lines}. As in \cite{TZ}, the proof is based on an induction on the degree $e$ of the rational curves.

\begin{proof}[Proof of \cref{lem:RationalCurves2Lines}]
Let $M \simeq \P^{ne + n + e}$ be the projective space of $(n+1)$-tuples of degree $e$ polynomials on $\P^1$, which we think of as a compactification of $\Mor_e(\P^1,\P^n)$. Define the following subset of $M$:
\begin{gather*}
  S_F = \set{(g_0(z_0,z_1),\dots,g_n(z_0,z_1)) \in M \vert \\
    F(g_0(z_0,z_1),\dots,g_n(z_0,z_1)) = (h(z_0,z_1))^2 \text{ for some } h(z_0,z_1) \in H^0(\sO_{\P^1}(ed))}.
\end{gather*}
Since the locus of square polynomials is closed, $S_F$ is closed.
From \cref{lem:RationalCurveLift},  if a point $\gamma \in S_F$ corresponds a morphism $\gamma \from \P^1 \to \P^n$, then there is a morphism $c \from \P^1 \to X$ such that $p \circ c = \gamma$, where $p \from X \to \P^n$ is the covering map.

We first need to study connectedness of $S_F$ after removing some of the points in $S_F$ that do not correspond to morphisms. Specifically, let $B$ be the set of $(n+1)$-tuples of degree $e$ polynomials such that the linear span is 1-dimensional. In symbols:
\[B \coloneqq \set{(c_0g,\dots,c_ng) \vert (c_0,\dots,c_n) \in \P^n,g \in \P(H^0(\P^1,\sO_{\P^1}(e)))}.\]
$B$ is a closed set of dimension $n+e$, corresponding to an $e$-dimensional choice of degree $e$ polynomial $g$ and an $n$-dimensional choice of a point in $\P^n$.

 We will now prove that $S_F \setminus B$ is connected using \cref{lem:ProjectiveConnectedness}. For any $(g_0,\dots,g_n) \in M$, we can write 
\begin{equation}
  \label{eq:FCoefficientForm}
  F(g_0,\dots,g_n) = \sum_{i=0}^{2de}a_ix_0^{2de-i}x_1^i,
\end{equation}
where the $a_i$ are polynomials in the coefficients of the $g_i$ \ie polynomials defined on $M$. Define the set
\[S_k \coloneqq \set{(g_0,\dots,g_n) \in M \vert a_0 = \cdots = a_{k-1} = 0, \sum_{i=k}^{2de} a_ix_0^{2de-i}x_1^i \text{ is a square}}, \]
and let $T_k$ be the set defined by $a_0 = \dots = a_k = 0$.
Note that $S_0 = S_F$, and that when $k$ is an even number, $S_k \cap T_k$ can be interpreted as the $(n+1)$-tuples $(g_0,\dots,g_n)$ such that $F(g_0,\dots,g_n)$ is the product of $x_0^{k+1}$ and a square polynomial of degree $\frac{2de-k-2}{2}$, hence $S_k \cap T_k = S_{k+2}$. 

From \cref{lem:SquareCondition} we see that for $j < de-1$, $S_{2j} \cup T_{2j}$ can be defined, as a set, by the vanishing of $de+j$ polynomials. To do this, we first require the $2j$ polynomials $a_0,\dots,a_{2j-1}$ to vanish. Then, using \cref{lem:SquareCondition}, we find $de-j$ polynomials in the $a_i$, $i = 2j,\dots,2de$, whose vanishing imply that $\sum_{i=2j}^{2de}a_ix_0^{2de-2j-i}x_i^{i-2j}$ is a square, or $a_{2j} = 0$. Together these $de+j$ homogenous polynomials define $S_{2j} \cup T_{2j}$. 

Similarly, using \cref{lem:QuadricSquarecondition} rather than \cref{lem:SquareCondition}, $S_{2de-2}$ can be defined by the vanishing of $2de-1$ homogeneous polynomials. Specifically, we take the $2de-2$ polynomials $a_0,\dots,a_{2de-3}$ and the polynomial $a_{2de-2}a_{2de}-a_{2de-1}^2$. Recall that we think of the $a_i$ as polynomials in the coefficients defining a morphism from $\P^1$.

It now follows from \cref{lem:ProjectiveConnectedness} that the sets $(S_{2j}\cup T_{2j}) \setminus B$ for $j = 1,\dots,de-2$ are connected. To see this set $N = ne + n + e$, $M = de+j$ and take $X$ to be $S_{2j}\cup T_{2j}$ and $Y$ to be $B$. Comparing $N-M-1$ and $\dim B$, we get
\begin{align*}
  N-M-1 &= ne + n + e - (de+j) - 1 \geq ne + n + e - (2de-2) - 1\\
 &=  (n-2d)e + n + e + 1 > n+e = \dim B.
\end{align*}
In the final inequality we use that $d < \frac{n}{2}$ and $e$ is non-negative.
For a similar reason, $S_{2de-2} \setminus B$ is connected, since
\begin{align*}
  ne + n + e - (2de-1) - 1 \geq (n-2d)e + n + e > n+e = \dim B.
\end{align*}

Since $(S_{2j}\cup T_{2j}) \setminus B$ is connected, and $(S_{2j}\cap T_{2j})\setminus B = S_{2j+2}$, we can apply \cref{lem:Topological}, in the ambient space $M \setminus B$, to show that if $S_{2j+2} \setminus B$ is connected, then so is $S_{2j}\setminus B$. Since we also know that $S_{2de-2} \setminus B$ is connected, we conclude that $S_{2j}$ is connected for all $j$. In particular, $S_0 \setminus B = S_F \setminus B$ is connected.

Now let $C$ in $X$ be a rational curve of degree $e \geq 2$, with a parametrization $a \from \P^1 \to C \subset X$. Our goal is to prove that $\im(a)$ is algebraically equivalent to a sum of rational curves of lower degree. Let $\alpha \from \P^1 \to \P^n$ be defined by $\alpha = p \circ a$. Pick any line $l_0 \subset X$, and let $l \subset \P^n$ be the image of $l_0$ by $p$. Let $\beta \in M$ correspond to the morphism $\beta \from \P^1 \to \P^n$ which is an $e$-fold cover of $l$ \ie a morphism $\P^1 \to l$ of degree $e$. Both $\alpha$ and $\beta$ are contained in $S_F$.

Since $S_F \setminus B$ is connected, we can find a connected chain of irreducible curves $\set{\Gamma_i}$, contained in $S_F$, between the points $\alpha$ and $\beta$. We may further assume that each pair of consecutive components meet in a single node.

Now let the set $D \subset M$ be the set of $(n+1)$-tuples $(g_0,\dots,g_n)$ that have a common factor, and therefore do not define a morphism. First assume that the nodes in the chain $\set{\Gamma_i}$ do not lie in $S_F \setminus D$. Then, after possibly deleting some points in $D$, we obtain a family of curves connecting the morphism $\alpha$ and the morphism $\beta$. This lifts to a chain of curves connecting the morphism $a$ with some morphism $b$, such that the composition $p \circ b$ is equal to $\beta$. But any such morphism must be an $e$-fold cover of one of the two lines in $p^{-1}(l)$. So the image of $a$ is algebraically equivalent to $e$ times the class of a line in $X$.

Finally, it remains to consider the case when one of the nodes in the chain $\set{\Gamma_i}$ lies in $D$. Approaching the first such node from the direction closest to $a$, we obtain a family of morphisms outside of $D$ approaching a point in $D$. By dividing out by the common factor, we see that this point in $D$ corresponds to a nonconstant morphism $\P^1 \to \P^n$ of degree strictly less than $e$. So in the Kontsevich space of stable maps, this means that the boundary stable map has at least one component of degree strictly less than $e$. Hence all components of the boundary stable map must have degree strictly less than $e$. So $C$ is algebraically equivalent to a union of rational curves of degree strictly less than $e$, and by induction on $e$ we can conclude that \cref{lem:RationalCurves2Lines} holds.
\end{proof}

\printbibliography[heading = subbibliography]
\stopcontents[chapters]