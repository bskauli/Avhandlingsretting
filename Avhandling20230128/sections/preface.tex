\chapter{Preface}


This thesis is submitted in partial fulfillment of the requirements
for the degree of \emph{Philosophiae Doctor} at the University of Oslo.
The research presented here was conducted at the University of Oslo,
under the supervision of professor John Christian Ottem, and cosupervised by professor Kristian Ranestad and associate professor Jørgen Rennemo.


The thesis consists of eight papers, preceded by an introduction summarizing their contents and placing them in their mathematical context. One paper is published and one is accepted for publication. The papers and are ordered thematically and are intended to be mostly self contained. However, there are some notable exceptions. In particular, \cref{pap:linesondoublecovers} serves as background material for \cref{pap:griffiths} and \cref{pap:coniveaudoublecovers}. Furthermore, \cref{pap:33diagonal} builds on the result in \cref{pap:23diagonal}. Finally, \cref{pap:coniveauhypersurfaces} is intended primarily as a complement to \cref{pap:coniveaudoublecovers}.


% \epigraph{Perhaps I could best describe my experience of doing mathematics in terms of entering a dark mansion. One goes into the first room, and it's dark, completely dark. One stumbles around bumping into the furniture, and gradually, you learn where each piece of furniture is, and finally, after six months or so, you find the light switch. You turn it on, and suddenly, it's all illuminated. You can see exactly where you were.}{Andrew Wiles}
\newpage
\section*{Acknowledgements}
%Joy of mathematics
%Advisor
%Referees of papers
%Struggle
%Friends and colleagues and Ingrid
%Family
Nearly a decade ago, I first arrived at the Department of Mathematics. At the time I was certain that I wanted to study physics and chemistry. However, after only a few months I was swept off my feet by the joy of mathematical thinking, and have never looked back. Now as my time at the Department of Mathematics is at an end, I wish to express my gratitude to everyone who has helped fill these years with both fun and learning.

First, I must thank my advisor John Christian Ottem for his patience with my many questions, and his encouragement and advice on my research. His comments on the papers in this thesis caught many embarrassing mistakes and improved the exposition greatly. I must also thank my co-supervisor Kristian Ranestad for his help, in particular his guidance on a research project that did not make it into this thesis.

I must also thank all my great teachers at the department for introducing me to the world of mathematics. It seems likely that at another university, the joys and wonders of this field would have remained alien to me, and my life would have been poorer as a result.

I am also grateful for the welcoming community of PhD students on the 11th floor. It has been a great pleasure to spend time with you, and I will miss both the mathematical and nonmathematical conversations we have had. In particular, I wish to thank Bernt Ivar, Martin, Håkon, Cédric, Elisa, Luca, nye Martin, Simen, Ola, Felix and Nikolai. Edvard deserves additional thanks for proofreading the introduction and providing helpful comments.

I have also received plenty of encouragement and welcome distractions from my friends outside of the Department of Mathematics. Looking back at the four years I have been working on the thesis, I see that they have been greatly improved by the many adventures during this time entirely unrelated to mathematics. Additionally, Ingrid deserves my heartfelt thanks for her kind words and patience, especially during the last months of writing. Her advice on punctuation rules, and keen eye for mistakes, was also greatly appreciated.

Most importantly, I must thank my family, and in particular my parents, Torbjørn and Kirsten, for their constant support. As with any worthwhile project, working on this thesis has come with its share of both ups and downs. Throughout it all I have relied on their kind encouragment. It seems plausible to me that without them, this thesis would never have been completed.

\vskip\onelineskip
\begin{flushleft}
    \sffamily
    \uiocolon\textbf{\theauthor}
\end{flushleft}
 